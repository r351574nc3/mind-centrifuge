\documentclass[11pt,a4paper,sans]{article}
\title{Commencement 2020}
\author{Leo Przybylski}
\begin{document}
\maketitle
\tableofcontents
\section{Prologue}

It is a great honor for me to be here; in this moment, to impart on you at the end of your milestone, a meager offering, such as it is. To strenghten, embolden, and inspire you as you go forth onto your next milestone. It really is a great honor. When I heard Michelle had asked me to do this, my first thought was, "I wonder how many people said, 'No'". Of course, I started on it immediately, July 02, 2020. Thank you to all for coming here to give your support and encouragement, to bear witness to the graduation of the class of 2019, and to observe the next generation to carry on our values, past, present, and future all before God. Now let us bow our heads as I lead a prayer.

\subsection{Prayer}
\begin{quote}
    Lord Father in Heaven, we humbly thank you for our children and our students. The joy and light that they have brought into our lives is dwarfed only by your own. We all praise you for your glorious creation that includes these small miracles. The pride we feel makes us all anxious with anticipating their promise of a future that you have made for them where you will use them and make them great beyond measure. As in any day, Lord, we entrust them and their futures to you. We pray to you this day that your blessings will be upon them. That you will continue to teach them the fruits of the spirit, and inform on them your will for their lives. We pray Lord that their eyes stay fixed on you forever and ever. Amen.
\end{quote}

\section{Introduction}

Before you embark, I am going to impart on you some of what has helped me when I stood where you are now. Some subtle reminders of who you are and where you come from, encouragement in the face of adversity, do's and don'ts I wish someone had told me, and other advice that you may soar like eagles. 


\section{Origins}

You have all learned what you need to elevate to the next milestone in your lives. You are ready. You would not be here if you were not ready for what lies ahead of you. It is easy when we are rigidly focused on what is in front us to lose sight of where we started. What makes all of this possible in the first place? Why are we here? What is driving us forward? Not surprising, the answer will help you in all circumstances.

Paul refered to himself as a bondservant. The reason is that Jesus paid the cost for our sins. As a result, we were bonded from sin. It was a free gift given to us, so nothing is really owed. Paul felt that he owed nonetheless. As Christians, we are the same in this respect. Paul felt the need to pay back the cost of his salvation by working for God. We should all feel this same way. 

\begin{quote}
    Whatever you do, work heartily, as for the Lord and not for men, 24 knowing that from the Lord you will receive the inheritance as your reward. You are serving the Lord Christ. --Colosians 2:23
\end{quote}

\section{Simple Acts}

Believe in simple acts. Many inspirational speakers will tell you about the importance of small goals and accomplishments that you can build on.

Since there are those among you making a committment to our Armed Forces this year, I felt it appropriate to include some anecdotes from Admiral William H. McRaven and Sun Tzu. Admiral McRaven has a poignant anecdote about the importance of making one's own bed. 

\begin{quote}
     If you make your bed every morning, you will have accomplished your first task of the day. It will give you a small sense of pride, and it will encourage you to do another task, and another, and another. By the end of the day, that one task completed will have turned into many tasks completed. Making your bed will also reinforce the fact that the little things in life matter. If you can't do the little things right, you'll never be able to do the big things right. If by chance, you have a miserable day, you will come home to a bed that is made. That you made. A made bed gives you encouragement that tomorrow will be better. --Admiral William H. McRaven
\end{quote}

Every day better than the day before. Tomorrow is a new day. Sometimes we can be our own worst critics. We let the day overcome us. 

\begin{quote}
    Be angry and do not sin; do not let the sun go down on your anger -- Ephesians 4:26
\end{quote}

This has meaning beyond just anger. Don't let the disappointment of yesterday ruin the promise of tomorrow. I am not saying to simply forget about the past. On the contrary, I believe this verse means to reconcile yourself with the past. Acknowledge that you cannot change what has been done. You can affect today and the next day.

\section{Humility vs. Self-deprecation}
I find myself often explaining the difference between these. One is healthy. The other is not. They both have power. One has the power to strengthen while the other serves only to weaken. One is a tool of God, the other is a weapon of the devil. One is built on seeing the truth of yourself, while the other is a lie about one's self. You can guess which is which. I think this distinction is important because when you start to talk a certain way, you also start to believe it. Others start to believe it as well.

\subsection{Self-deprecation}
This is essentially the diminishing of one's self to anyone. That's it. If ever you diminish yourself, this is self-deprecation. Examples:
\begin{itemize}
    \item I will never be as great as \ldots
    \item I am a bad \ldots
    \item My \ldots are bad.
\end{itemize}

\subsection{Humility}
Humility is knowing who you are and what you are capable of and what you are not capable of past, present, and future. I say past, present, and future because humility means knowing the progress you have made. Acknowledging your growth. It also means acknowledging your potential and hope for the future. This will come in later, but hope is powerful. 

It is not boasting you are greater than you are and it is not bragging about your inadequacies either. 

\begin{quote}
    A man's gotta know his limitations -- Hairy Callahan
\end{quote}

Humility is being able to not just admit, but to own that we are not perfect. Because we know we are each not perfect, the expectation of perfection is implicitly omitted. It does not mean we are 

\begin{quote}
 And he sat down and called the twelve. And he said to them, “If anyone would be first, he must be last of all and servant of all. -- Mark 9:25
\end{quote}
Examples, of humility:
\begin{itemize}
    \item I can still improve on\ldots
    \item There are better people than I for this.
    \item I could always learn more about \ldots
    \item I can use the experience in \ldots
\end{itemize}

\section{Responsibility}
There are several facets to responsibility. There's being responsible, taking responsibility, and having responsibilities. When you take responsibility, you tend to be given responsibilities. The reason is that leadership recognizes this and will relegate responsibility where it is most appreciated. In 2007, University of Arizona was planning to implement a new financial system. They pulled me in as advisor because of my experience with the product. I gave them instructions, trained, and gave advice on a path forward. I was very much on the outside on this one. They seemed to be making some decent progress. Eventually, it came to the point where they needed to demonstrate some of that progress which ended in failure. In a meeting, it was asked, "What led the the problems and how can we fix them?" Those that answered seemed to pass-the-buck in an effort to blame our operations team which did not appreciate that. That was especially the case since the initiative was being managed by the operations team leader. I stepped up. I didn't lie. I told it how it was. Our team was ill-prepared. The training they were given was not enough. They did not have experience in the processes required to complete the tasks. The communication was not well-coordinated because there was no plan ahead of time. Everything was manual which opened the door to human error. Even if there wasn't a mistake this time, there were bound to be mistakes later. The meeting adjourned immediately after my remarks. In the next hour, they promoted the previous team-leader to manager, and put me in charge of the team going forward. I was given autonomous control over the entire process. They basically said, "Do what you need to do to fix this." It wasn't the proudest moment in my career, but what I learned most was
\begin{itemize}
    \item Leadership took notice that I observed what went wrong and had already adjusted my plan to deal with it. They didn't care who's fault it was. They cared how they were going to make it right. 
    \item There is no shame in defeat, only shame in failing to learn from it.
    \item Taking responsibility can also be seizing opportunity. In taking responsibility for the problem, I was also seizing responsibility over the solution.
    \item Failing upward is a real thing.
\end{itemize}

\begin{quote}
    Victory comes from finding opportunities in problems. – Sun Tzu
\end{quote}

\section{Be Bold}
Be bold. Do not fear what lies ahead.

I want to say something about fear. The Bible teaches us the importance of fearing God, but that is not the kind of fear I am talking about. I am talking about the kind of fear that forces you to doubt. The kind of fear that is itself a lie. There are a lot of things to be afraid of in this world, but we do not have the time to go over them. 

\subsection{Failure}

Here is an anecdote from Admiral McRaven. 
\begin{quote}
    Every day during training, you were challenged with multiple physical events. Long runs, long swims, obstacle courses, callisthenics. Something designed to test your metal. Every event had standards (times you had to meet). If you failed to meet those times (those standards), your name was posted on a list. At the end of the day, those on the list were invited to a circus. A circus was two hours of additional callisthenics designed to wear you down, to break your spirit, to force you to quit. No one wanted a circus. A circus meant that for that day, you did not measure up. A circus meant more fatigue, and more fatigue meant that the following day would be more difficult and more circuses would be likely. At some time during S.E.A.L. training everyone everyone made the circus list. An interesting thing happened to those constantly on the list. Over time, those students who did two hours of extra callisthenics got stronger and stronger. The pain of the circus built inner strength and physical resiliency. Life is full of circuses. --Admiral William H. McRaven
\end{quote}

Failure is inevitable. At the risk of sounding cliche, success is built on a mountain of failures. If I am choosing players for my sports team, I am going to choose a team of players that have never lost. Not because I am worried about their sportsmanship. Rather, I want people on my team that no exactly what not to do. Undefeated cannot tell me that because they have never lost.

\begin{quote}
    Victorious warriors win first and then go to war, while defeated warriors go to war first and then seek to win. – Sun Tzu
\end{quote}

Fail fast. Once you have accepted and reconciled failure is inevitable, get it done. Get it done quickly. The faster you fail, that faster you will succeed. 


\begin{quote}
    If you know the enemy and know yourself you need not fear the results of a hundred battles. – Sun Tzu
\end{quote}

Take us for example: Andrea just reminded me earlier that if by some chance I manage to really mess this up, I might have a chance to fix it 4 years from now. 

\subsection{Disappointment}

Fear of disappointment is the fear that you will not measure up to the expectations of those that came before you. In some cases, you may even be fearful to disappoint yourself or even God. Know this. God put you on this path. You are not here except for His knowing of what you are capable of. He knows your failures. He knows your successes. You are here for no one else but Him.

\begin{quote}
     Have I not commanded you? Be strong and courageous. Do not be frightened, and do not be dismayed, for the Lord your God is with you wherever you go. -- Joshua 1:9
\end{quote}

\subsection{Boldness}

\begin{quote}
    The wicked flee when no one is pursuing them, the righteous are as bold as a lion. -Proverbs 28:1
\end{quote}

\begin{quote}
    Be strong and courageous. Do not fear or be in dread of them, for it is the Lord your God who goes with you. He will not leave you or forsake you. -- Deuteronomy 31:6
\end{quote}


\section{Hope}

Hope is powerful. Hope will get you to make your bed in the morning. It will make you fight the most unwinnable battles. It will make you face your greatest failures and disappointments. It will help you face uncertain days. 

\begin{quote}
    And not only so, but we glory in tribulations also: knowing that tribulation worketh patience; And patience, experience; and experience,  hope: -- Romans 5:4
\end{quote}

\begin{quote}
    Rebellions are built on hope. -- Cassian Andor
\end{quote}


\section{Conclusion}
I have now quoted "The Art of War", "Star Wars", "The Holy Bible", "Dirty Hairy", and The US Navy. Mission accomplished. In a nutshell, we are all bonded to God. Whatever we do, we do for Him and not for ourselves. He has a vested interest in your futures. At some point in your lives, you will fail. You will fall short of your goals. Remember that God is with you. He will give you the strength and resolve to push forward.

One more thing. Be non-conformists. If the world is telling you one thing, seek God for the answers.

\begin{quote}
  I beseech you therefore, brethren, by the mercies of God, that ye present your bodies a living sacrifice, holy, acceptable unto God, which is your reasonable service. And be not conformed to this world: but be ye transformed by the renewing of your mind, that ye may prove what is that good, and acceptable, and perfect, will of God.   -- Romans 12:1-2
\end{quote}


\end{document}

%\section{Fruits of the Spirit}
%\begin{quote}
%And now these three remain: faith, hope, and love; but the greatest of these is love.
%    
%\end{quote}
%
%\subsection{Faith}
%\subsection{Hope}
%\subsection{Love}

%Christianity is not your religion and God is not your diety. God is your Creator and Holy Father. The Holy Spirit He sent to you as a Counselor. Jesus Christ, His Son, is your Savior. Christianity is who you are. You are Christians. Little Christs. That is who you are. Do not forget it. If you do, I will find you and set you straight.